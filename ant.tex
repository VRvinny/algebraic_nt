\documentclass[11pt]{article}

\usepackage[left=3cm, right=3cm, top=3cm, bottom=3cm]{geometry}
\usepackage{amsmath}
\usepackage{amsfonts}
\usepackage{amsthm}
\usepackage{ragged2e}
%crossed out elements
\usepackage{cancel}
\usepackage{amssymb}
%lightning symbol
\usepackage{wasysym}
%position tables
\usepackage{placeins}
%use right curly braces
%\usepackage{mathtools}
%place text under sigmas and pi
\usepackage{mathtools}
%kappa%
\newenvironment{rcurly}{\left.\begin{aligned}}{\end{aligned}\right\rbrace}

\begin{document}

\numberwithin{equation}{subsection}
\newtheorem{theorem}{Theorem}[section]
\newtheorem{defn}[theorem]{Definition}
\newtheorem{prop}[theorem]{Proposition}
\newtheorem{cor}[theorem]{Corollary}
\newtheorem{lemma}[theorem]{Lemma}
\newcommand{\R}{\mathbb{R}}
\newcommand{\Z}{\mathbb{Z}}
\newcommand{\C}{\mathbb{C}}
\newcommand{\Q}{\mathbb{Q}}
\newcommand{\A}{\alpha}
\newcommand{\ring}{\mathcal{O}}
\newcommand{\spa}{$ $\\[-0.5em]}
\newcommand{\spac}{$ $\\[1em]}
\newcommand{\B}{\beta}
\newcommand{\G}{\gamma}
\newcommand{\bba}{\mathcal{B}}
\title{Algebraic Number Theory}
\author{Vinesh Ramgi}
	%\date{}\o
	\maketitle


\newpage
	\begin{abstract}

		\centering{What did the number theorist say as he drowned?}\\[1em]
		Log, log, log, log....\\[2em]

		For an up to date version of this pdf, check my GitHub :)\\[0.5em]
https://github.com/vrvinny/algebraic\_nt

	\end{abstract}
\newpage
\tableofcontents{}
\newpage
	\section{Introduction/Review}
	\subsection{Introduction}
		This is the study of certain rings, numbers called algebraic integers, e.g. 
	
	\begin{itemize}
		\item{Quadratic rings $\Z[\sqrt{d}]$ or $\Z [\frac{1+\sqrt{d}}{2}$]}
		\item{Cyclotomic rings $\Z[\zeta_n] \implies y = e^{2\pi \i /n }$}
		\item{$\Z[z\sqrt{2}] = \{ x+y^3\sqrt{2} + z^3 \sqrt{z}^2 \}$ $x,y,z \in \Z$}
	\end{itemize}


	\begin{defn}
		A Diophantine equation is an equation of the form $f(x_1, \dots, x_n)=0 $ where $f$ is a polynomial with coefficients in $\Z$ 
	\end{defn}

	We'll usually be interested in solution in integers (or maybe rational numbers), for example, Pell's equation- $x^2 - dy^2 = 1$, or $N(A) = n $ where $A= x + y \alpha $, $\alpha = \{ \sqrt{d},  \frac{1+\sqrt{d}}{2}\}$ .

	In general Diophantine equations are hard, Matiyasevich's theorem shows Diophantine equations are as hard as any mathematical question.
	Inspite of this, there are some Diophantine eqations for which we have methods for solving, e.g. \textit{What are the integer solutions of $x^3 = y^2 + y = y(y+1) ? $}.
	$ $\\[1em]
	Since  $y,y+1$ are both coprime and their product is a cube, both $y$ and $y+1$ are a cube which implies $y=0,-1$. So we have two solutions, $(0,0), (0,-1)$. To do this we used this lemma:


	\begin{lemma}
		Descent Lemma $ $

		Let $R$ be a ring be a unique factorisation domain. Suppose $a,b,c \in R$ and $a^n = bc$. If $b,c$ are coprime in $R$ then $b=u^{rn}$, $ c=vs^n$ where $u,v$ are units in $R$.
	\end{lemma}

Another example, $x^3 = y^2 + 1$:

	Problem, $y^2+1$ doesn't factorise in $\Z$ but it does factorise in $\Z[i] \implies x^3 = (y+i)(y-i)$. We want to use the Descent lemma to solve the equation.
	\begin{itemize}
		\item{$\Z[i]$ is a unique factorisation domain}
		\item{Are $y+i$ and $y-i$ coprime in $\Z[i]$? }
	\end{itemize}
	Suppose $p \in \Z[i]$ is an irreducible common factor of $y+i$, $y-i$. If $p | y+i$ and $p|y-i \implies p|(y+i) - (y-i) \implies p | 2i$. This means the norms also divide each other, $N(p) | N(2i) \implies N(p) | 4 $.

	$N(p) \neq \pm 1$ because $p$ isn't a unit, therefore,  $ 2|N(p)|N(y+i) \implies N(y+i) = y^2+1 = x^3$, so $2| x^3$.

	Since $2$ is a prime, $2|x \implies x^3 \equiv 0 \hspace{7pt} (8)$. This implies $y^2 + 1 \equiv 0\hspace{7pt} (8) $


\begin{center}

\begin{tabular}{l|l}
\hline
$y$       & $y^2 \mod 8 $ \\ \hline
0       & 0             \\ 
$\pm 1$ & 1             \\ 
$\pm 2$ & 4             \\ 
$\pm 3$ & 1             \\ 
4       & 0             \\ 
\end{tabular}

\end{center}
Since the equation has no solutions, $y^2 + 1 \equiv 0 \hspace{7pt} (8)$ has no solutions. 
$ $\\
So $y+i, y-i$ are coprime in $\Z[i]$.


$\therefore y+1 = uA^3$ with $u \in \Z[i]^\times $, $A \in \Z[i]$ with $u=\pm 1 $ or $\pm i$
$ $\\[1em]
So in fact
\begin{align*}
	y+i &= (r+si)^3 \hspace{7pt} r,s \in \Z \\
	& = r^3 + 3ir^2s - 3rs^2 - is^3\\
	& = r^3 - 3rs^2 + i(3r^2s - s^3)
\end{align*}
Organising the terms gives a new Diophantine equation:
\begin{align*}
	1= 3r^2s - s^3 && y = r^3 - 3rs^2
\end{align*}
We can solve the new equation: 
$ $\\[1em]
$1= (3r^2-s^2)s$ $\implies $ $s= \pm 1$
$ $\\[1em]
If $s=1 \implies 3r^2 - 1 =1 $ \lightning\\
If $s=-1 \implies 3r^2 - 1 = -1 \implies r =0 $ \hspace{7pt} so $(r,s) = (0,-1)$
$ $\\[1em]
This implies $x=1, y= 0$ so $(1,0)$ is the only solution in integers.

$ $\\[1em]
This motivates the question: which rings are unique factorisation domains? 
More specifically which rings of algebraic integers are unique factorisation domains?

\subsection{Definitions and proofs}
A ring is a set $R$ with two operations $ + $ \& $  \times $. 
$(R, +)$ is an abelian group with identity element $0$. 
$\times $ is commutative, associative and has identity 1. 
$x(y+z) = xy+xz$ $\forall x,y,z \in R$
$ $\\[1em]
An element $x \in R$ is:
\begin{itemize}
	\item{a unit if $\exists $ $x^{-1} \in R $ such that $xx^{-1} = 1$ }
	\item{reducible if $x=yz $ where $y,z$ not units}
	\item{irreducible otherwise}
\end{itemize}
For examples, in $\Z$, units are $\pm1$, irreducible elements are $\pm p $ for prime numbers $p$.

\begin{defn}
	A ring $R$ is an integral domain if $xy=0 \implies x=0$ or $y=0$
\end{defn}
\begin{lemma}
	Cancellation property: If $R$ is an integral domain, if $x \neq 0$ then $xy=xz \implies y=z$
\end{lemma}
\begin{proof}
	\begin{align*}
	xy =xz &\implies x(y-z) = 0 \text{ and since } x\neq 0\\
		&\implies y-z =0 \\
		&\implies y=z
	\end{align*}
\end{proof}
$ $\\
A ring $R$ is a unique factorisation domain if:
\begin{itemize}
	\item{$R$ is an integral domain}
	\item{If $x \in R$ and $x \neq 0 $ then $x=Up_1 \dots p_r$  with $U \in R^\times $ and $P_i$ irreducible }
\end{itemize}
$ $\\
Suppose $p_1 \dots p_r  = q_1 \dots q_s$ with $p_i, q_i$ irreducible then $r=s$ and we can renumber so that $q_i = Up_i$ with $U \in R^\times$.

The $3^{rd}$ condition is equivalent to if $p \in R$ is irreducible and $p|ab $ then $p| a$ or $p | b$.

\begin{lemma}
	Descent Lemma: Let $R$ be a UFD (Unique Factorisation Domain) and let $a,b,c \in R$ with $a^n = bc $ and $b,c$ coprime.
	Then $b = ur^n$ and $c= vs^n$ with $u,v \in R^\times$
\end{lemma}


\begin{proof}
If $a$ is a unit then $b $ and $c$ are units, so the result is true.

If $a=0$ then $b=0$ or $c=0$.

WLOG assume $b=0 = 1*0^n$. 
	But $b$ and $c$ are coprime $\implies c$ must be a unit (since it is a common factor of $b$ and $c$.

	In other cases $a = p_1 \dots p_r $ with $p_i $ irreducible. 
	So 
	\begin{align*}
		b &=  \text{(a unit)} * p_1^{s_1} \dots p_r^{s_r} && s_i + t_i = n \forall i\\
		c &= \text{(a unit)} * p_1^{t_1} \dots p_r^{t_r} 
	\end{align*}
But we're assuming $b,c$ are coprime so:\\
$\implies $ each $s_i $ is either 0 or $n$ \\
	$\implies  b = $ (a unit) * (some $n^{th}$ power)  
	$\implies  c = $ (a unit) * (some $n^{th}$ power)  
\end{proof}
$ $\\
\textbf{Reminder about quadratic rings}\\
Let $d\neq 1$ be a square free integer and $\Z[\alpha] = \{ x+y\alpha : x,y \in \Z \}$
	\begin{equation*}
\alpha = \alpha_d = 
	\begin{cases}
		\sqrt{d} & d\not \equiv 1 \hspace{7pt} (4) \\
		\cfrac{1+\sqrt{d}}{2} & d \equiv 1 \hspace{7pt} (4)
	\end{cases}
	\end{equation*}

If $A = x+y\sqrt{d}$ then $\bar A = x-y \sqrt{d} $ and $N(A) = A \bar A = x^2 - dy^2 $.

Similarly $N(x+y\frac{1+\sqrt{d}}{2}) = x^2 +xy+ \frac{1-d}{4}y^2$.
$ $\\[1em]
The ring $\Z[\alpha]$ is norm-Euclidean if for all $A, B \in \Z[\alpha]$ iwht $B \neq 0$, $ \exists Q, R \in \Z[\alpha]$ such that $A=QB+R$ and $|N(R)| < |N(B)|$.

\begin{prop}
	If $\Z[\alpha]$ is norm-Euclidean then $\Z[\alpha]$ is a UFD.
\end{prop}
\begin{proof}
	(Sketch)\\
	We'll show that if $p \in \Z[\alpha]$ is irreducible and $p|AB$ then $p|A$ or $p|B$.\\
	If $p \not | A $ then $ hcf(A,P) = 1 \implies 1 = HA+KP$ by the Euclidean algorithm.\\
	$B = \underbrace{HAB}_\text{multiple of $p$} +\underbrace{KPB}_\text{multiple of $p$}$ $\implies p | B$
\end{proof}

\begin{theorem}
	Disappointing Theorem: $\Z[\alpha_d] $ is norm-Euclidean in the following cases $d = 2,3,5,13,-1,-2,-3,-7,-11$.
	Conjecturally there are infinitely many real quadratic rings which are UFD.
\end{theorem}

\section{Materials from other courses on rings, ideals and fields}
Mainly from Galois Theory, Commutative Algebra and Groups and Rings. 
In this course, 

\begin{itemize}
	\item{All rings are commutative with 1}
	\item{A field is a ring with $1 \neq 0$}
	\item{If $x\neq 0$, $x$ is a unit }
\end{itemize}

\begin{defn}
An ideal $I$ in a ring $R$ is a subset of $R$ such that:

	\begin{itemize}
		\item{$(I.+)$ is a subgroup of $R$}
		\item{$\forall x \in R $, $y \in I$ $\implies xy \in I$}
	\end{itemize}
\end{defn}
$ $\\
Examples: If $x \in R$ then we define $(x) = \{ xy: y \in R \}$.
This set $(x)$ is an ideal in $R$.


\begin{defn}
	Ideals of the form $(x)$ are called principal ideals. 
	$(x)$ is the principle ideal generated by $x$.\\
	$(x_1, \dots, x_n) = \{x_1y_1 + \dots + x_ny_n : y_i \in R\}$ this is also an ideal.
\end{defn}
$ $\\[1em]
\textbf{E.g.} in $\Z$ $(4,6) = (2)$ and in general $(n,m) = (hcf(n,m))$

\begin{defn}
A principal ideal domain is an integral domain such that all ideals are principal. A Noetherian ring is a ring in which ideals are finitely generated.
\end{defn}
$ $\\[1em]
To show that $(x_1, \dots, x_n) \subseteq I$, it is equivalent to showing $x_1, \dots, x_n \in I$
$ $\\
\textbf{E.g.} in $\Z$ $(4,6) = (2)$
\begin{proof}
	\begin{align*}
		(4,6) \subseteq 2 && (2) \subseteq(4,6) \\
		4 = 2*2 \in (2) && 2=2*4 + (-1) *6 \in (4,6)\\
		6 = 2*3 \in (2) 
	\end{align*}
\end{proof}


\begin{defn}
A principal ideal domain is an integral domain in whicch every ideal is principal
\end{defn}

\textbf{E.g.} $\Z$ is a PID 

\begin{proof}
	Suppose $I \subseteq \Z $ is an ideal.\\[0.5em]
	If $I=\{0\} $ then $I=(0)$.\\
	If $I \neq \{0\}$ choose $x \in I$ with $|x|$ as small as possible with $x\neq 0 $.\\[1em]
	Claim $I=(x) $, $x \in I \implies x \subseteq I$.\\[1em]
Conversely, suppose $y \in I$ with $ y = qx + r$ such that $|r| < |x| $.
	This means that $r = y-qx \in I$ $\implies r = 0 \implies y= qx \in (x) $ which means $I\subseteq (x)$.
\end{proof}
$ $\\[0em]

\textbf{E.g.} If $\Z[\alpha]$ is a norm-Euclidean quadratic ring then $\Z[\alpha]$ is a PID.
\begin{proof}
	Replace $|x|$ by $|N(A)$ for $A \in \Z[\alpha]$.
\end{proof}

 \textbf{E.g.} If $k$ is a field then $k[x]$ is a PID.
\begin{proof}
	Replace $|x|$ by $deg(f)$ for $f \in k[x]$.
\end{proof}
\subsection{Quotient Rings}
Let $I$ be an ideal in $R$, we'll say $x=y \in (I)$ if $x-y \in I$.
$ $\\[1em]
$R/I = \{$Congruency classes of elements of $R$$\}$

\begin{lemma}
	If $x \equiv x'$ $(I)$ and $y \equiv y'$ $(I)$ then $x+y  \equiv x'+y'$ $(I)$ and $xy = x'y'$ $(I)$
\end{lemma}
This means we can make $R/I$ into a ring.

\begin{proof}
If $x-x' \in I$ and $y-y' \in I$ then:
	\begin{align*}
		(x+y) - (x'+y')  &= (x-x') + (y-y') \in I \\[2em]
		xy - x'y' &= xy - xy' + xy' -x'y'\\
		&= x(y-y') + (x-x')y \in I
	\end{align*}
\end{proof}
$ $\\[-2em]
\textbf{E.g.} If $R = \Z $, $ I= (n) $ and  $x \equiv x' (I) \Leftrightarrow  x \equiv x' \hspace{7pt}(n) \implies R/i = \Z/n $ 
\\[1em]
\textbf{E.g.} If $k$ is a field an $f \in k[x]$ with degree $d$ and $I = (f)$, every element of $k[x]$ is congruent to a unique polynomial of degree $<d$. (The remainder after dividing by $f$).
$ $\\ 
$\therefore$  $R/I = \{ a_0 + a_1 + \dots + a_{d-1}x^{d-1} : a_i \in k \}$

\begin{defn}
Let $R$ and $S$ be rings. 
A ring homomorphism is a function $\phi : R \rightarrow S$ such that:
	\begin{itemize}
		\item{$\phi(x+y) = \phi(x) + \phi(y)$}
		\item{$\phi(xy) = \phi(x) + \phi(y)$}
		\item{$\phi(1_R) = 1_S$}
	\end{itemize}
	$ $\\
	$ker(\phi) = \{ x \in R : \phi (x) =0 \}$ and $ker(\phi)$ is an ideal of $R$ (trivial to prove) \\
	$im(\phi) = \{ \phi(x) : x \in R\}$ and $im(\phi)$ is a subring of $S$
\end{defn}


\subsection{1st Isomorphism Theorem for Rings}

Let $\phi : R \rightarrow S$ be a ring homomorphism. 
Then there is an isomorphism:
\begin{align*}
	R/ker(\phi) \cong im(\phi) \text{ by the mapping } (x\mod ker(\phi)) \rightarrow \phi(x)
\end{align*}

\subsection{Maximal Ideals}

\begin{defn}
Let $R$ be a ring.
An ideal $M \subseteq R$ is maximal if:
	\begin{itemize}
		\item{$M \neq R$}
		\item{If $M \subseteq I \subseteq R $ with $I$ an ideal then $I=M $ or $R$}
	\end{itemize}
\end{defn}


\begin{prop}
	Suppose $R$ is a PID, an ideal $(x)$ is maximal if and only if $x$ is irreducible.
\end{prop}
\begin{proof}
Algebra 4
\end{proof}
$ $\\[-1em]
Note that the difference between a field and a ring is that if $x \neq 0$ then $xx^{-1} = 1$ and $1 \neq 0$.
\begin{prop}
In any ring $R$, $M$ is maximal if and only if $R/M$ is a field
\end{prop}
\begin{proof}
	$(\implies)$ Assume $M$ is maximal therefore $M \neq R$ and $1 \not \in M$ and $1 \not \equiv 0 \hspace{7pt} (M)$.


	Assume $x \not \equiv 0 \hspace{7pt}(M)$, let $I$ be the ideal generated by $M$ and $x$, then $ I \supsetneq M$ which means $I=R$.
	$ $\\[1em]
	$I = \{ m + xy : m \in M, y \in R \} \implies 1 = m + xy $ and $ 1 \equiv xy \hspace{7pt}(M)$
	$ $\\[1em]
	($\Longleftarrow$) Assume $R/M$ is a field so $1 \not \equiv 0 \hspace{7pt} (M)$.

	$\therefore 1 \not \in M$ so $M \neq R$.

	Suppose $M \subsetneq I$, want to show $I=R$.
	Choose $x \in I, x \not \in M $ so $x \not \equiv 0 \hspace{7pt }(M)$
	$ $\\[0.5em]
	So $\exists y $ such that $xy = \equiv 1 \hspace{7pt} (M) $ and by the definition of existence of $x^{-1}$, $1 \in I \implies I = R$
\end{proof}


\begin{cor}
	Let $k$ be a field and $f \in k[x]$, then $k[x]/(f)$ is a field if $f$ is irreducible over $k$
\end{cor}
\begin{proof}
	$f $ irreducible $\iff (f) $ is maximal $ \iff k[x]/(f) $ is a field.\\
	(Polynomial ring is ideal)
\end{proof}

\subsection{Field extensions}
\begin{defn}
If $k $ and $l$ are fields with $k \subseteq l$ then $k$ is a subfield of $l$. 

$l$ is called a field extension of $k$, $e.g.$ $\R \subseteq \C$ 
\end{defn}
When $l$ is an extension of $k$, we can think of $l$ as a vector space over $k$.

$ $\\
\textbf{E.g.} $\C$ has basis $\{1,i\}$ as a vector space over $\R$

The degree of the extension $[l:k]$ is the dimension of $l$ as a vector space over $k$ \\
\textbf{E.g.} $[\C : \R] = 2$ \\
\textbf{E.g.} $\Q(i) = \{x+iy : x,y \in \Q \}$

This is an extension of $\Q$ with basis $\{1,i\} $ $\implies [\Q(i) : \Q] = 2$
$ $\\[0.5em]
\textbf{E.g.} Let $f \in \Q[x] $ be irreducible $\implies \Q[x] /(f) = \{a_0 + a_1x + \dots a_{d-1}x^{d-1} \} $ is a field.

This is an extension of $\Q$ and has degree $d=deg(f)$.

$\{ 1,x,\dots, x^{d-1} \}$ is a basis, so $[\Q[x]/(f) :\Q] = d = deg(f)$.

$ $\\
\textbf{Notation: } Let $l$ be an extension of $k$ and let $\alpha \in l$, then:

\begin{defn}
	$\alpha$ is called "algebraic over $k$" if there exists a non-zero $f \in k[x] $ such that $f(\alpha) = 0$.
	Otherwise $\alpha$ is transcendental.
\end{defn}
$ $\\[-0.5em]
\textbf{E.g.} $\sqrt{2} $ is algebraic over $\Q$ since it is a root of $x^2 - 2$
$ $\\[1em]
For any $\alpha \in l$ $k[\alpha] = \{ g(\alpha) : g \in k[x]\}$, the ring generated by $k$ and $\alpha$.
$ $\\
$k(\alpha) = \{ \cfrac{g(\alpha)}{h(\alpha)} : g,h \in k[x], h(\alpha) \neq 0\}$, the field generated by $k$ and $\alpha$.


\begin{prop}
	Let $\alpha$ be algebraic over $k$.
	Then there is a unique monic polynomial $m(x) \in k[x]$ such that:
	\begin{itemize}
		\item{$m(\alpha) = 0 $}
		\item{$f(\alpha) = 0 \iff m | f$}
	\end{itemize}
\end{prop}

$m$ is the only monic irreducible polynomial over $k$ such that $m(\alpha) = 0$.


\begin{defn}
This polynomial is called the minimal polynomial of $\alpha$ over $k$
\end{defn}
$ $\\[-0.5em]
\textbf{E.g.} $i$ is algebraic over $\R$ with minimal polynomial $x^2 + 1$

$i$ is algebraic over $\Q$ with minimal polynomial $x^2 + 1$

$i $ is algebraic over $\C$ with minimal polynomial $x-i$


\begin{cor}
	Let $\alpha$ be algebraic over $k$, then $k[\alpha] = k(\alpha)$, i.e. $k[\alpha]$ is a field and there is an isomorphism 
	\begin{align*}
		k[x]/ (m) &\cong k(\alpha)\\
		(g(x) \mod m) &\mapsto g(\alpha)\\
		a_0+a_1x +\dots + a_{d-1}x^{d-1} &\mapsto a_0 + a_1\alpha + \dots +a_{d-1}\alpha^{d-1} 
	\end{align*}
	where $m$ is the minimal polynomial and has degree $d$.
\end{cor}

$\{ 1, \alpha, \dots, \alpha^{d-1} \}$ forms a basis for $k(\alpha)$ and $[k(\alpha) : k] = d = deg(m)$




\begin{proof}
	We have a homomorphism (surjective) $k[x] \to k[\alpha] \implies k[x] / (m) \cong k[\alpha]$ and $g \to g(\alpha) $.
	(Field because $m$ is irreducible)

	Kernel $= \{ g: g(\alpha) = 0 \} = (m)$, therefore $k[\alpha]= k(\alpha) $ the isomorphism takes $(g(x) \mod m)$ to $g(\alpha)$.
	$ $\\[1em]
	 $ \{ 1,x,\dots, x^{d-1} \}$ is a basis for $k[x]/(m) $
	 $ $\\[1em]
	 So $ \{ 1,\alpha, \dots, \alpha^{d-1} \}$ is a basis for $k(\alpha)$
\end{proof}
$ $\\[-0.5em]
\textbf{E.g.} $i$ is algebraic over $\R$ with minimal polynomial $x^2 + 1$


$ \R[x] /(x^2 + 1) \cong \R(i) = \C$ with the map $a+bx \mapsto a+bi $ $a,b \in \R$

$ $\\[1em]
Similarly $\Q[x] /(x^2 + 1) \cong \Q(i) $ with $a,b \in \Q$ with the map $ a+bx \mapsto a+bi$
$ $\\[1em]
\textbf{E.g.} $\alpha = \sqrt[3]{2}$ This is a root of $x^3 - 2$ and so $\alpha$ is algebraic (over $\Q$).

$\Q[x] / (x^3 -2 ) \cong \Q(\alpha) $ with the mapping $a+bx+cx^2 \mapsto a+b\alpha + c\alpha^2$.

The degree of the extension $[\Q(\alpha) : \Q] = 3$ with $\{1,\alpha, \alpha^2 \}$ a basis for $\Q(\alpha)$ over $\Q$.

\subsection{Finding Minimal Polynomials}
We need methods to check that a polynomial is irreducible over $\Q$.

\begin{lemma}
	Gauss Lemma: Suppose $f \in \Z[x]$ and $f=gh \hspace{7pt} g,h \in \Q[x] $, then there exists $c \in Q^\times $ such that $cg$ and $c^{-1} $ are in $\Z[x].$
\end{lemma}

\begin{lemma}
	Monic Gauss Lemma: Let $f \in \Z[\alpha]$ be monic, if $f=gh \in \Q[x]$ both monic then $g,h \in \Z[x]$.
\end{lemma}
\begin{cor}
	If $f \in \Z[x] $ is monic the $f$ is irreducible over Q $\iff $ irreducible over $\Q$
\end{cor}

\begin{cor}
	Let $f \in \Z[x]$ be monic.
	Let $\bar f$ be the reduction of $f \mod n $, i.e. $\bar f \in (\Z / n)[x].$
	If $\bar f$ is irreducible then $f$ is irreducible over $\Z$ and over $\Q$.
	Note that $n$ doesn't need to be prime
\end{cor}


\subsection{Eisenstein's Criterion}
 Let $f \in \Z[x]$ and let $p$ be prime.
Let $f(x) = a_dx^d+ \dots + a_0$, if $p \not | a_d $ and $f(x) \equiv a_dx^d \hspace{7pt} (p)$ and let $ f(0) \not \equiv 0 \hspace{7pt} (p^2) $ then $f $ is irreducible over $\Z/p^2 $ and over $\Q$.

$ $\\
\textbf{E.g.} $\alpha = 10^\frac{1}{11} \implies \alpha^{11} = 10$.


$\alpha $ is a root of $x^{11} - 10$

$x^{11} -10 $ is irreducible by Eisenstein's Criterion (either with $p=2$ or $p=5$)

So $m_\alpha(x) = x^{11}-10 $ 
$ $\\[1em]
\textbf{E.g.} $\alpha = 2^{\frac{2}{3}}$

$\alpha^3 = 4$

$\alpha$ is a root of $m(x) = x^3 -4$

To show that $m$ is the minimal polynomial, we must show that $m$ is irreducible.

$m(x+1) = x^3 + 3x^2 +3x-3$

$m(x+1)$ is irreducible by Eisenstein's criterion with $p=3 \implies m(x) $ is irreducible.
$ $\\[1em]
\textbf{E.g.} $\alpha = 3^{\frac{2}{3}} $

$\alpha^3 - 9 = 0$

$\alpha$ is a root of $m(x) = x^3 -9 $.
Note that $deg(m) = [\Q[\alpha):\Q]$.
To show that $m$ is the minimal polynomial it's sufficient to show that $[\Q(\alpha) : \Q] \geq 3$

$\alpha = 3\beta$ where $\beta = 3^{\frac{1}{3}} = \frac{1}{3}\alpha^2$ so $\beta \in \Q(\alpha) \implies \Q(\beta) \subseteq \Q(\alpha)$
$ $\\[-0.5em]

$\beta $ has minimal polynomial $x^3 - 3$ (by Eisenstein's criterion)
$ $\\[-0.5em]

$\therefore [\Q(\beta):\Q] = 3$
$ $\\[-.5em]

$\therefore m(\alpha)$ has degree $\geq 3$
$ $\\[-.5em]

$\therefore m_\alpha = x^3 -9$ 
$ $\\[1em]
Alternatively suppose $x^3 -9$ factorises over $\Q$.
By the Monic Gauss lemma, $x^3 -9 = (x-a)(x^2 +bx+c)$.
By comparing coefficients, $ac=9$, which means $a = \pm 1$ or $\pm 3$ or $\pm 9$ and $a^3 = 9$ because $a$ is a root. \lightning
$ $\\[1em]
\textbf{E.g.} $ \alpha = \sqrt{2} + \sqrt{3} $
\begin{align*}
	\alpha^2 & = 2+2\sqrt{6} +3 && (\alpha^2 - 5) -24 = 0\\
	& = 5 + 2\sqrt{6}
\end{align*}

So $\alpha $ is a root of $m(x) = x^4 - 10x^2 + 1$

Suppose $m(x) = (x-a)(x^3+bx+^2+cx+d) $ with $a,b,c,d \in \Z$
$ $\\

$a $ is a root of $m$ and $a$ is a factor of $m(0) = 1$ so $a = \pm 1$ which is a contradiction since $m(\pm 1) = -8 \neq 0.$

The other possible factorisation is $m(x) = (x^2 + ax+b)(x^2+cx+d)$.
Comparing coefficients:
\begin{align*}
	0 &= a+c && \\
	-10 &= b+d+ac && c = -a &&a^2 = 10 \pm 2 = 8 \text{ or } 12 \text{ \lightning} \\ 
	0 &= ad+bc\\
	1 &= bd && b=d=\pm 1\\
\end{align*}


\subsection{Roots of Unity}
Let $n$ be a positive integer. 
An $n^th$ root of unity is a complex number $\zeta$ such that $\zeta^n = 1$.
A primitive $n^{th}$ root of unity is an $n^{th}$ root of unity which is not a $d^{th}$ root of unity for any $d<n$.


$n^{th}$ root of unity: $e^{2\pi i \frac{a}{n}}$ for $a = 0,1,\dots,n-1$

Primitive $n^{th}$ root of unity $e^{2\pi i \frac{a}{n}}$ for $a \in (\mathbb{Z}/n)^\times$.

There are $\phi(n)$ primitive $n^{th}$ roots of unity.
$ $\\[1em]
The $n^{th} $ cyclotomic polynomial is $\Phi_n(x) = \underbrace{\prod}_{\substack{\text{Primitive $n^{th}$} \\ \text{ roots of unity }}} (x - \zeta)$

$deg(\Phi_n(x)) = \phi(n)$





\begin{lemma}
	For any $n$: $\prod_{d|n} \Phi_d(x) = x^n - 1$.
	We can use this to calculate $\Phi_n$.
\end{lemma}
$ $\\
\textbf{E.g.} If $p$ prime then $\Phi_1(x)\Phi_p(x) = x^p -1 $.

So $\Phi_p(x) = \cfrac{x^p-1}{x-1} = 1+x+\dots + x^{p-1}$



\begin{cor}
	$\Phi_n(x)$ has coefficient in $\Z$.
\end{cor}
$ $\\[-0.5em]
\textbf{Remark:} If $\zeta$ is a primitive $n^{th}$ root of unity then $\zeta$ is a root of $\Phi_n$
\begin{theorem}
Each $\Phi_n$ is irreducible over $\Q$.
	So $\Phi_n$ is the minimal polynomial of a primitive $n^{th}$ root of unity.
\end{theorem}


\subsection{Tower Law}

\begin{theorem}
	Suppose $k \subseteq l \subseteq m $ be fields. Then $[m:k] = [m:l]*[l:k]$
\end{theorem}

\begin{proof}
	Sketch: Let $\{ b_1, \dots, b_n \}$ be a basis for $l$ as a vector space over $k$.

	Let $\{c_1, \dots, c_m \}$ be a basis for $m$ over $k$.

	Then $\{ b_i, c_j\}$ s a basis for $m$ over $k$.
\end{proof}
$ $\\
\textbf{E.g.} $\Q \subseteq \Q(\sqrt{2}) \subseteq \Q(\sqrt{2},\sqrt{3}) $
$ $\\[-0.5em]

$[\Q(\sqrt{2}) : \Q] = 2 $ and $x^2 - 2$ is the minimal polynomial.
$ $\\[-0.5em]

The minimal polynomial of $\sqrt{3} $ over $\Q(\sqrt{2})$ is a factor of $x^2 -3$ so degree is 1 or 2.

From the Tower theorem, we know $[\Q(\sqrt{2},\sqrt{3}):\Q] = \cancel{2}$ or $4$.
$ $\\[-.5em]

$\alpha = \sqrt{2} + \sqrt{3} \in \Q (\sqrt{2}, \sqrt{3})$
$ $\\[-.5em]

$\Q \underbrace{\subseteq}_{\substack{\text{deg 4 since } \\ \text{$\alpha$ has} \\ \text{ minimal} \\ \text{polynomial}\\ \text{$x^4 -10x^2 + 1$} }} \Q(\sqrt{\alpha}) \subseteq \Q(\sqrt{2},\sqrt{3}) $
$ $\\

By the Tower law $[\Q(\sqrt{2},\sqrt{3}): \Q]$ is a multiple of 4.
$ $\\[-.5em]

$ \therefore [\Q(\sqrt{2},\sqrt{3}): \Q] = 4$
$ $\\[-.5em]

$ \therefore [\Q(\sqrt{2},\sqrt{3}): \Q(\sqrt{2}] = 2$
$ $\\[-0.5em]

$x^2 - 3$ is irreducible over $\Q(\sqrt{2})$
$ $\\[-0.5em]

Also $ [\Q(\sqrt{2},\sqrt{3}): \Q(\alpha)] = 1$ i.e. $\Q(\sqrt{2},\sqrt{3}): \Q(\alpha)$

$ $\\[-.5em]
This is an example of:

\subsection{Primitive Element Theorem}
\begin{theorem}
	Suppose $k$ is an extension of $\Q$ of finite degree then there exists $\alpha \in k$ such that $k = \Q(\alpha)$
\end{theorem}

\begin{proof}
	Sketch: $k$ has only finitely many subfields.
	Choose an $\alpha \in k$ which is not in any of the proper subfields.
	$ $\\
	
	$\Q (\alpha)$ is not contained in a proper subfield of $k$ but $\Q(\alpha)$ is a subfield of $k$, therefore $\Q(\alpha) = k$.
\end{proof}

\subsection{Conjugates and Complex Field Embeddings}

Let $\alpha, \beta $ be algebraic numbers.
Then $\alpha$ and $\beta$ are conjugates if $m_\alpha = m_\beta$. They have the same minimal polynomial over $\Q$.
$ $\\[1em]
\textbf{E.g.} $\alpha = i$, $\beta = -i$ both have minimal polynomial $x^2 + 1$.\\
\textbf{E.g.} $\zeta = e^{2 \pi i \frac{1}{100}} $ This is a conjugate of $\zeta^3$, both have minimal polynomial $\Phi_{100}$.\\
\textbf{E.g.} $\sqrt{2} + \sqrt{3}$ has conjugates $\pm \sqrt{2} \pm \sqrt{3}$


\subsection{Galois Separability Lemma}
\begin{lemma}
	If $\alpha$ is algebraic over $\Q$ then $m_{\alpha}(x)$ has no repeated roots in $\C$, i.e. $\alpha$ has exactly $d$ conjugates in $\C$ where $d = deg(m(x)) = [\Q(\alpha) : \Q]$
\end{lemma}


\begin{proof}
	Suppose $(x-\beta)^2$ $|$ $ m_\alpha(x) $ where $m_\alpha$ is the minimal polynomial of $\beta$ over $\Q$.

	If $(x-\beta)$ $ | $ $ m_\alpha$ then $\alpha - \beta $ $ | $ $ m_\alpha '(x) $ but $m_\alpha'$ has smallest degree than $m_a$. \lightning


\end{proof}

\begin{defn}
	An algebraic number field is an extension $k \supseteq \Q$ with $[k: \Q]$
\end{defn}
$ $\\[-0.5em]
By the primitive element theorem $k = \Q (\alpha)$ for some $\alpha \in k$ so $k \cong \Q[x] /(m) $ when $m$ is the minimal polynomial of $\alpha$ and $ deg(m) = [k: \Q]$.
The polynomial has exactly $d$ roots in $\C$ where $d= deg(m)$. 
Call these roots $\alpha_1, \alpha_d$. 
Each $\A_i$ has minimal polynomial $m$ over $\Q$.
$ $\\[1em]
$\therefore \Q[x] /(m) \cong \Q(\A_i) \subseteq \C$
$ $\\[-0.5em]

So for each conjugate $\A_i$ of $\A $ in $\C$, there is a field homomorphism:
\begin{align*}
	\sigma_i : k &\to \C\\
	 k &\to \Q[x]/(m) \to \Q(\A_i) \subseteq \C\\
\end{align*}
\vspace{-3em}
\begin{align*}
	a_0 +a_1\A + \dots + a_{d-1} &\to& a_0+a_1x_1+\dots+a_{d-1}x^{d-1} &\to& a_0+a_1\A_1 + \dots + a_{d-1}\A_i^{d-1} 
\end{align*}


$ \sigma_i (a_0 + \dots + a_{d-1} \A^{d-1}) = a_0 +\dots + a_{d-1}\A_i^{d-1}$
\begin{prop}
$\sigma_1, \dots, \sigma_d$ are all the field homomorphisms from $k $ to $\C$
\end{prop}
$ $\\[-0.5em]
\textbf{E.g.} $k = \Q(\sqrt{2})$

The conjugates of $\sqrt{2}$ in $\C$ are $\pm 2$. 
$\A_1 = \sqrt{2}$,
$\A_2 = -\sqrt{2}$
\begin{align*}
	\sigma_1(x+y\sqrt{2}) = x +y\sqrt{2} \\
	\sigma_2(x+y\sqrt{2}) = x - y\sqrt{2}
\end{align*}

$k = \Q(\A)$ , $m(x) = x^3 - 2$ and $\A_1 = 2^{\frac{1}{3}}$, $\A_2 = 2^{\frac{1}{3}}e^{\frac{2\pi i}{3}} ,\A_3 = 2^{\frac{1}{3}} e^{\frac{4\pi i }{3}}$

\newpage
\section{Rings of Algebraic Integers}

\begin{defn}
	Let $k$ be a field extension of $\Q$.
	$\alpha \in k$ is an algebraic number if $f(\alpha)=0 $ for some $f \in \Q[x]$.
\end{defn}
$ $\\[-.5em]
$\A$ is an algebraic integer if $f(\A) = 0 $ for some $f \in \Z[x]$.
$ $\\[1em]
\textbf{E.g.} $\sqrt{2}$ is an algebraic integer $f(x) =x^2 -2$\\
\textbf{E.g.} $\alpha = \frac{1+\sqrt{5}}{2}$
\begin{align*}
	\bigg(\A - \frac{1}{2}\bigg)^2  - \frac{5}{4} &= 0\\
	\A^2 - \A -1 & =0\\
\end{align*}

So $f(x) = x^2 - x-1$
$ $\\[1em]
\textbf{E.g.} $\zeta = e^{\frac{2\pi i}{n}}$ then $f(x) = x^n - 1 \implies \zeta$ is an algebraic integer.

\begin{prop}
	Suppose $\alpha$ is an algebraic number. 
	Then $\A$ is an algebraic integer $\iff$ $m_\A \in \Z[x] $ (minimal polynomial).
\end{prop}
$ $\\[-0.5em] 
\textbf{E.g.} $\frac{1}{\sqrt{2}}$ is not an algebraic integer, it's minimal polynomial is $x^2 - \frac{1}{2}$

\begin{proof}
	($\Longleftarrow$) if $m_\A(x) \in \Z[x] $ then $\A$ is an algebraic integer $m_\A(\A) = 0$ and $m_\A \in \Z[x]$ is
	
	monic.
$ $\\
	$(\implies)$ Assume $\alpha$ is an algebraic integer $f(\alpha) = 0 $, $f \in \Z[x]$ monic.

	$m_\A  | f $ in $\Q(x) $ by definition of $m_\A$, i.e. $f = m_\A^\times q$ , $m_\A, q \in \Q[x] $ monic.

	By the Monic Gauss lemma, $m_\A$ and $q \in \Z[x]$. 
\end{proof}

\begin{cor}
The algebraic integers in $\Q$ are $\Z$
\end{cor}
\begin{proof}
Let $\A \in \Q $.
	$m_\A(x) = x-\A $.
	$m_\alpha \in \Z[x] \iff \A \in \Z$.
\end{proof}
$ $\\[-0.5em]
Main aim of this chapter: Given an algebraic number field $k$, what are the algebraic integers in $k$?
$ $\\[1em]
\textbf{Notation}: $\ring_k = \{ $Algebraic integers in $k\}$.\\
\spa
\textbf{E.g.} $\ring_\Q = \Z$ If $\A \in Q$ then $m_\A(x) = x-\A$



\begin{prop}
Let $A$ be an $n * n$ matrix with entries in $\Z$.
Then all eignevalues of $A$ are algebraic integers.
\end{prop}
\begin{proof}
	The eigenvalues are roots of $Ch_A(x) = \det(xI_n-A)$
\end{proof}
\spa
\textbf{Notation}:
$ \Z [ \A ] = \{f(\A) : f \in \Z[x] \}$ the ring generated by $\A$
\spac
As a group $\Z[\A]$ is generated by $\{1,\A,\A^2 , \dots \}$, i.e. $\Z[\A] = Span_\Z\{1,\A,\A^2 \dots \}$
\spac
Sometimes $\Z[\A]$ is finitely generated as an additive group, i.e. $\Z[\A] = Span_\Z\{b_1, \dots, b_r\}$.

\spa
\textbf{E.g.} $\Z[\sqrt{2}] = Span_\Z\{b_1, \dots, b_r\}$


\begin{prop}
	$\Z[\A]$ is finitely generated as an additive group $\iff \alpha$ is an additive integer.
\end{prop}

\begin{proof}
	$(\implies)$ Assume $\Z[\A] = Span_\Z\{b_1, \dots, b_r \}$ with $b_i \in \Z[\A]$
	\begin{align*}
		\A = a_{i1}b_1 + \dots + a_{1r}b_r && \text{ for some } a_{ij} \in \Z
	\end{align*}
	\begin{align*}
		\alpha b_1 = a_{11}& + \dots a_{1r}b_r \\ 
		&\vdots&& 	\begin{pmatrix}
b_1\\
\vdots\\
b_r
		\end{pmatrix}
		 = 
		 (a_{ij}) 
		\begin{pmatrix}
b_1\\
\vdots\\
b_r
		\end{pmatrix}
\\
		\alpha b_r = a r_1& + \dots a_{rr}b_r
	\end{align*}
	\begin{align*}
		\A
	\end{align*}
	So $\A$ is an eigenvalue of $(a_{ij})$ and $\A$ is an algebraic integer.
	\spac $(\Longleftarrow)$ Assume $\A$ is an algebraic integer $f(\A)=0$ for some monic $ f \in \Z[x]$.
	
	Let $d=deg(f)$, claim $\Z[\A] = Span_\Z\{1,\A, \dots, \A^{d-1} \}$.

	Let $R$ be a ring, $f,g \in R[x]$ such that $g=qf+r$ for $q,r \in R[x]$ with $deg(r)<d$
\spa

	$\Z[\A]=\{g(\A : g \in \Z[x]\}$
	\spa

	$g = qf=r$, \hspace{7pt} $deg(r) < d$, \hspace{7pt} $q,r \in \Z[x]$
\spa

	$g(\A) = q(\A)f(\A) + r(\A)$
	\spa

	$g(\A) = r(\A) = a_0 + a_1\alpha + \dots + a_{d-1} \A^{d-1}$
	\spa

	$g(\A) \in Span_\Z \{ 1, \A, \dots, \alpha^{d-1} \} $
\end{proof}

\begin{cor}
Let $\A, \beta \in k$ be both algebraic integers, then $\A + \beta$, $\A -\beta$, $\A\beta$ are algebraic integers form a subring of $k$.
	\begin{proof}
	Let $\Z[\A] = Span_\Z \{ 1,\A, \dots, \A^{d-1} \}$
		
		\hspace{37pt}$\Z[\beta]= Span_\Z \{ 1,\beta, \dots, \beta^{e-1}\}$
		\spa

		$\A^i * \beta^j \in Span_\Z \{ \A^i \beta^j:  i<d, j<e \}$
		\spa
		
		$\therefore \Z[\A,\beta] = Span_\Z \{\A^i \beta^j : i<d, j<e\}$
\spa

		So $\Z[\A, \beta]$ is finitely generated. 
\spa

		Let $\gamma = \A + \beta$ or $\A-\beta$ or $\A\beta $ so $\gamma \in \Z[\A,\beta]$
		\spa
		
		$\Z[\gamma] \subseteq \Z[\A,\beta] \implies \Z[\gamma]$ is finitely generated as an additive group.
		\spa

		$\implies \gamma$ is an algebraic integer.
	\end{proof}
\end{cor}
$\spa$
\textbf{E.g.} Let $k = \Q(\sqrt{5}) = \{ x+y\sqrt{5} : x,y \in \Q\} $. What is $\ring_k$?
\spa

Let $A = x+ y\sqrt{5}$, with $ x,y \in \Q$, $y\neq 0$.

Assume $A \in \ring_k $, $ k=\Q(A)$ and $m_A(x)$ has degree 2.
\spa

$(A-x)^2 - 5y^2 = 0$
\spa

$A^2 -2xA - 5y^2 + x^2=0$
\spa

So $m_A(x) = X^2-2xX+x^2 -5y^2$
\
\begin{align*}
	A \in \ring_k &\iff m_A(x) \in \Z[x]\\
	&\iff 2x \in \Z, \text{ and }x^2-5y^2 \in \Z
\end{align*}

This means the denominator of $x$ is at most 2.

\begin{align*}
	x&= \frac{r}{2} \hspace{7pt}r \in \Z \text{ since }x^2-5y^2 \in \Z \\[1em]
	y&= \frac{s}{2} \hspace{7pt} s \in \Z
\end{align*}
\begin{align*}
	\frac{r^2 - 5s^2}{4} \in \Z \implies r^2-5s^2 &\equiv 0 \hspace{7pt} (4)\\
	r^2 &\equiv s^2 \hspace{7pt}(2)\\
	r &\equiv s \hspace{7pt} (2)\\
	r = s& +2t \hspace{7pt} t \in \Z
\end{align*}
\begin{align*}
	A &=\frac{r}{2}+\frac{s}{2} \sqrt{5}\\[0.5em]
	& =\frac{s}{2} + t + \frac{s}{2}\sqrt{5}\\[0.5em]
	& = t + s\bigg(\frac{1+\sqrt{5}}{2}\bigg)
\end{align*}

So every algebraic integer in $k$ is actually in $\Z[\A]$ where $\A = \frac{1+\sqrt{5}}{2}$.

Conversely since $\A \in \ring_k$ we know $\Z[\A] \subseteq \ring_k$. 
$\ring_k = \Z[\A] = \{t+s\A : t,s \in \Z \}$.
\spac
What is so special about the rings $\ring_k ?$
Why not just study $\Z[\sqrt{5}]$ instead of $\Z[\frac{1+\sqrt{5}}{2}]$?



\begin{prop}
	Suppose $R \subseteq k $ is a UFD subring of $k$. If $A = \frac{\A}{\beta} $ is an algebraic integer in $k$ with $\A, \beta \in R$ then $A \in R$.
\end{prop}
\spa
	This implies $\Z[\sqrt{5}]$ is not a UFD (take $A = \frac{1+\sqrt{5}}{2})$.

\begin{proof}
	Let $A = \frac{\A}{\beta}$ be an algebraic integer. 
	Without loss of generality assume, $\A, \beta \in R$ are coprime.

	To prove $A \in R$ we'll show that $\beta $ is a unit.

	Suppose $p | \beta $, $p \in R$ is irreducible.
 	\spa

	$A \in \ring_k$ so $A^d + \A_{d-1}A^{d-1} + \dots a_0 = 0$ \hspace{7pt} $a_i \in \Z$


	\begin{align*}
		\frac{\A^d}{\B^d}  + a_{d-1} \frac{\A^{d-1}}{\B^{d-1}} + \dots + a_0 = 0\\
	\end{align*}

	Multiply by $\beta^d$:
	\begin{align*}
	\A^d + \underbrace{a_{d-1}\A^{d-1}\B + \dots + a_0\B^d}_\text{ multiples of $p$} = 0
	\end{align*}

	So $ p | \A^d$ and $p | \A$ since $R$ is a UFD and $p$ is irreducible.
	
	$p$ is a common factor of $\A $ and $\B$. \lightning 
	$\B$ has no irreducible factors so $\B$ is a unit.

\end{proof}
	\subsection{The Standard Represenation}

	\begin{defn}
The map $\B \mapsto A_\B$ is called the standard representation.
	\end{defn}
\spa
	Let $k / \Q$ be an algebraic number field.

	$d = [k: \Q]$ for some $\A \in k$
	\spa

	$k = \Q (\A)$
	\spa

	$m = m_\A$ minimal polynomial of $\A$
	\spa

	$\A_1, \dots, \A_d \in \C$ the (distinct) complex roots of $m_\A$, i.e. conjugates of $\A$ in $\C$.
	\spac
	For each $\A_i$ we have a field embedding:
	\begin{align*}
		\sigma_i&: k \longrightarrow \C\\
		\sigma_i&(a_0 + a_1\A + \dots + a_{d-1}\A^{d-1} ) = a_0 + a_1\A_i + \dots + a_{d-1}\A_i^{d-1}\\
		\sigma_i&(f(\A)) = f(a_i) \hspace{7pt} f\in \Q[x]
	\end{align*}
\spa
	For an element $\B \in k$, let $A_\B$ be the linear map $k \rightarrow k$, $A_\B(x) = \B x$.
	After choosing a basis for $k/\Q, $ $A_\B$ is a d$\times$d matrix of rational numbers.
	\spac
\textbf{E.g.} $k = \Q(\sqrt{2})$, $\B = 3+2\sqrt{2} $

We find the matrix of $A_\B$ with respect to the basis $\{1, \sqrt{2}\}$.
\begin{align*}
A_\B (1) = (3+2\sqrt{2})*1 = 3\cdot1 + 2\sqrt{2}\\
	A_\B (\sqrt{2}) = (3+2\sqrt{2})*1 = 4\cdot1 + 3\sqrt{2} &&
A = \begin{pmatrix}
	3 & 4\\
	2 & 3\\
\end{pmatrix}
\end{align*}
\spa
\textbf{Remark} : We can recover $\B$ from $A_\B$ by $\B = A_\B(1)$ and the map $\B \mapsto A_\B$ is injective.

\subsubsection{Properties of the Standard Representation}

\begin{itemize}
	\item{$A_{\B+\gamma} = A_{\B} + A_{\gamma}$ \hspace{7pt} $(\gamma, \B \in k \hspace{7pt} x \in \Q )$} 
	\item{$A_{\B\gamma} = A_{\B} \cdot A_{\gamma}$ }
	\item{$A_{x\B} = xA_{\B}$ }
\end{itemize}


\begin{proof}
	\begin{align*}
 A_{\B\gamma}(t) &= \B\gamma t\\
		&= A_\B(\gamma t)\\
		&= A_\B(A_\gamma(t) )&& \text{so } A_{\B\gamma} = A_\B \circ A_\gamma \\
	\end{align*}
Other proofs are similar.
\end{proof}

\begin{cor}
	If $g \in \Q[x]$ \hspace{7pt} $ A_{g(\B)} = g(A_\B)$. 
	The polynomial = the linear map
\end{cor}

\begin{proof}
	Let $g(x) = a_0 + a_1x + \dots + a_nx^n $\hspace{7pt} $a_i \in \Q$. 
	Then:
	\begin{align*}
		A_{g(\B)} &=A_{a_0 + a_1\B+\dots+a_n\B^n}\\
		&= A_{a_0} + A_{a_1\B} + \dots + A_{a_n\B^n}\\
		& = a_0A_1 + a_1A_\B + \dots + a_nA_{B^n}\\
		&= a_0A_1 + a_1A_\B+1_2(A_\B)^2 + \dots + a_n(A_\B)^n\\
		&= g(A_\B)
	\end{align*}
\end{proof}
\begin{defn}
	Let $\B \in k $ be the field polynomial $F_\B(x)$ is $F_\B(x) = \det(xI - A_\B)$ i.e. the characteristic polynomial of $\B$.
\end{defn}
\spa
$F_\B$ is the monic polynomial of degree $d$ in $\Q[x]$.

\begin{lemma}
	If $k= \Q(\A)$ then $F_\A(x) = m_\A(x)$.
\end{lemma}

\begin{proof}
	By the Cayley-Hamilton ($Ch_A(A) = 0$): $F_\A(A_\A) = A_{F_\A(\A)} = \underline{0}$.
	So $ F_\A(\A) = 0$ and $F_\A$ is a multiple of $m_\A$.
	Since it has degree $d$ and is monic, $F_\A = m_\A$.
\end{proof}

\begin{theorem}
	For any $\B \in k$, the matrix $A_\B$ is diagonalisable over $\C$ with diagonal entries $\sigma_1(\B), \dots, \sigma_d(\B)$ and $F_\B(x) = \big(x-\sigma_1(\B)\big) \dots \big(x-\sigma_d)\big)$
\end{theorem}
\begin{proof}
	First prove in the case $\B = \A$, $k = \Q(\A))$.

	The eigenvalues of $A_\A$ are the roots of $F_\A = m_\A$ (by the lemma)
	\spa

	These are $\A_1, \dots, \A_d$.
	By the Galois Separability lemma, we have $d $ eigenvalues.
	\spa

	$\therefore A_\A$ is diagonalisable over $\C$ and $P^{-1} A_\A P =$ 
	$
	\begin{pmatrix}
		\A_1 & &0\\
		 &\ddots &  \\
		0 && \A_d
	\end{pmatrix}
	$
	\spa

	Now check for any $\B \in k $ \hspace{7pt} $\B=g(\A)$ for some $g \in \Q[x]$

	\begin{align*}
	A_\B = A_g(\A) &= g(A_\A)a\\
		&= g \Bigg( P
		\begin{pmatrix}
			\A_1 & &\\
			& \ddots &\\
			&&\A_d
		\end{pmatrix}
		P^{-1}
		\Bigg)\\
		&= P g\Bigg(
		\begin{pmatrix}
			\A_1 & &\\
			& \ddots &\\
			&&\A_d
		\end{pmatrix}
		\Bigg) P^{-1}\\
		& = P
		\begin{pmatrix}
			g(\A_1) & &\\
			& \ddots &\\
			&&g(\A_d)
		\end{pmatrix}
		P^{-1} \\
		&= P
		\begin{pmatrix}
			\sigma_1(\B) & &\\
			& \ddots &\\
			&&\sigma_d(\B)
		\end{pmatrix}
		P^{-1}
	\end{align*}
\end{proof}
\spa

\textbf{E.g.} $k = \Q(\sqrt{2}) = \{ x+y\sqrt{2} : x,y \in \Q \}$ \hspace{7pt} $d=2$.
\spa

Use the basis $\{ 1, \sqrt{2} \}$
\begin{align*}
	A_{\sqrt{2}} (1) &= \sqrt{2}\cdot 1  = 0 \cdot 1 + 1\cdot \sqrt{2}\\
	A_{\sqrt{2}} (\sqrt{2}) &= 2\cdot 1 + 0\cdot \sqrt{2} &&
	A_{\sqrt{2}} = 
	\begin{pmatrix}
		0 & 2\\
		1 & 0 
	\end{pmatrix}
\end{align*}
\spa

Let $P = 
\begin{pmatrix}
	\sqrt{2} & \sqrt{2}\\
	1 &1
\end{pmatrix}
$ and
$
P^{-1}A_{\sqrt{2}}P=
\begin{pmatrix}
	\sqrt{2} & 0 \\
	0 & -\sqrt{2}
\end{pmatrix}
$
\spac

Let $\B = x+y\sqrt{2}$ \hspace{7pt}  $A_\B = x I_2 + y$ \hspace{7pt} $A_{\sqrt{2}} 
\begin{pmatrix}
	x &2y\\
	y&x\\
\end{pmatrix}
$
\spac

$
P^{-1}A_{\B}P=
\begin{pmatrix}
	x+y\sqrt{2} & 0\\
	0 &x-y\sqrt{2}
\end{pmatrix}
=\begin{pmatrix}
 \sigma_1(\B) &\\
	& \sigma_2(\B)
\end{pmatrix}
$
\spa

$F_\B(x) = \big(x-\sigma_1(\B)\big)\big(x-\sigma_2(\B)\big)$



\begin{cor}
	For any $\B \in k$, $[k : \Q(\B)]$
\end{cor}
\begin{proof}
	Suppose $p(x)$ is a monic irreducible factor of $F_\B$ in $\Q[x]$.
	Want to show $p=m_\B$.
	\spa

	The roots of $F_\B$ are $\sigma_1(\B), \dots, \sigma_d(\B)$.
	These are the conjugates of $\B$, i.e. the roots of $m_\B$.
	\spa

	Therefore the roots of $p$ have minimal polynomial $m_\B$.
	\spa

	$m_\B | p$ so $p = m_\B$ and since $F_\B $ and $m_\B$ are both monic, $F_\B = m_\B^r$ for some $r \in \Z$.
	Also $m_\B$ has degree $[\Q(\B) : \Q]$.
\spa

	So by the Tower law, $r = [k: \Q(\B)]$.
\end{proof}
\begin{cor}
	$\B \in \ring_{k} \iff F_\B(x) \in \Z[x]$
\end{cor}
\begin{proof}
	$(\implies)$
\spa

	Suppose $\B \in \ring_k $, $m_\B \in \Z[x]$ but $F_\B $ is a power of $m_\B \implies F_\B \in \Z[x]$
\spac
	$(\Longleftarrow)$
	Assume $F_\B \in \Z[x] $ \hspace{7pt} $ F_\B(\B) = m_\B(\B)^{[k:\Q(\B)]} = 0^{Y} = 0$ for some y.
	Therefore $\B \in \ring_k.$
	
\end{proof}

\textbf{E.g.} $k = \Q(i) $\hspace{7pt} $ \B = \frac{3}{2} + \frac{5}{7}i$
\begin{align*}
	F_\B(x) &= \big(x-\sigma_1(\B)\big)\big(x-\sigma_2(\B)\big)\\
	& = \bigg(x- \frac{3}{2} - \frac{5}{7} i\bigg)\bigg(x-\frac{3}{2} + \frac{5}{7}i\bigg)\\
	& = \bigg(x-\frac{3}{2}\bigg)^2 + \frac{25}{49} \not \in \Z[x] && \text{ so } \B \not \in \ring_k
\end{align*}

\begin{cor}
	Let $b \in k$ then $n \in \Z$ $(n > 0)$ such that $n \B \in \ring_k$.
\end{cor}

\begin{proof}
	Choose $n$ so that $nA_\B = A_{n\B}$ has entries in $\Z$. 
	Then $F_{n\B} \in \Z[x]$ and so by the previous corollary $n\B \in \ring_k$. 
\end{proof}
\spa
\textbf{E.g.} $k = \Q(\A)$, $\A = 10^{\frac{1}{3}}$,\hspace{7pt}  $\A \in \ring_k$ so $\Z[\A] \subseteq \ring_k$.
\spa

$\Z[\A] = \{x+y\A+z\A^2\ : x,y,z \in \Z \}$
\spa

Let $\B = \frac{1+\A + \A^2}{3}$ then $ A_\B  = \frac{1}{3} \begin{pmatrix}
	1 & 10 & 10\\
	1 & 1 & 10\\
	1& 1 & 1
\end{pmatrix}$
Is $\B$ an algebraic integer?
\spa

$F_\B(x) = \det(xI_3 - A_\B)$
\spa 

First calculate the standard representation:

\begin{align*}
	A_\A(1) & = \A = 0\cdot 1 + 1\cdot \A + 0\cdot\A^2\\
	A_\A(\A)& = \A^2 = 0\cdot 1 + 0\cdot \A + 1\cdot\A^2\\
	A_\A(\A^2)& = 10 =  10\cdot 1 + 0\cdot \A + 0\cdot\A^2\\ &&&&&
\end{align*}

$A_\A = \begin{pmatrix}
	0 & 0 & 10\\
	1 & 0 & 0 \\
	0 & 1 & 0
\end{pmatrix}$
$A_\A^2 = \begin{pmatrix}
	0 & 0 & 10\\
	1 & 0 & 0\\
	0 & 1 & 0
\end{pmatrix}^2 = 
\begin{pmatrix}
	0 & 10 &0\\
	0 & 0 &10 \\
	1 &0 &0 
\end{pmatrix}
$
\spac

$A_\B =\frac{1}{3} \Bigg ( I_3 + 
 \begin{pmatrix}
	 0 &0 &10\\
	 1 &0 &0\\
	 0 &1 &0\\
 \end{pmatrix}
 +
 \begin{pmatrix}
	 0 &10 &0\\
	 0 &0 &10\\
	 1 &0 &0\\
 \end{pmatrix}
\Bigg)
= \frac{1}{3}
 \begin{pmatrix}
	 1 &10 &10\\
	 1 &1 &10\\
	 1 &1 &1\\
 \end{pmatrix}
$
\spac

\begin{align*}
	F_\B &= \det(xI_3 - A_\B)\\
	& = \det\Bigg(\frac{1}{3}
	\begin{pmatrix}
		3x-1 & -10 & -10\\
		-1 & 3x-1 & -10\\
		-1 & -1 & 3x-1
	\end{pmatrix}
	\Bigg)\\
	&=\frac{1}{27} \bigg( \big( (3x-1)^3 -100-10) - \big(3(10(3x-1))\big) \bigg) \\
	& = \frac{1}{27} \big(27x^3 - 27x^2 -81x-81 \big)
\end{align*}

$F_\B(x) = x^3 -x^2 -3x-3 \in \Z[x] $ \hspace{7pt} $\B \in \ring_k$
\spa

$\Z[\A]]$ is not a UFD.



\subsection{Norms and Traces}

\begin{defn}
	For an element $\B \in k $ we define $N(\B) = \det(A_\B)$ to be the norm of $\B$ and $Tr(\B) = Tr(A_\B)$ to be the trace of $\B$.
\end{defn}

Note that $N(\B)$ and $Tr(\B) \in \Q$.
\spac 
\textbf{E.g.} $Tr(\begin{pmatrix}
	3 & 7 & 3\\
	4 & 7 & 6\\
	1 & 2 & 3
\end{pmatrix} = 3 + 7+ 3$

\begin{prop}
	$N(\B) = \sigma_1(\B) \times \dots \times \sigma_d(\B)$\\
	$Tr(\B) = \sigma_1(\B) + \dots + \sigma_d(\B)$ 
\end{prop}
\begin{proof}
	$A_\B \thicksim \begin{pmatrix}
		\sigma_1(\B) & & 0\\
		&\ddots &\\
		0 && \sigma_d(\B)
	\end{pmatrix}$
\end{proof}

\subsection{Properties of Norms and Traces}

\begin{itemize}
	\item{$N(\B\G) = N(\B) N(\G) $ for $\B, \G \in k$}
	\item{$N(x\B) = x^dN(\B)$ for $x \in \Q$}
	\item{$Tr(\B+\G) = Tr(\B) + Tr(\G)$}
	\item{$Tr(x\B) = xTr(\B)$}
\end{itemize}

\begin{proof}
	$N(\B\G) = \det(A_{\B\G}) = \det(A_\B A_\G) = N(\B) N(\G)$
\end{proof}

\begin{prop}
	$F_\B(x) = x^d - Tr(\B) x^{d-1} + \dots + (-1)^d N(\B)$
\end{prop}
\begin{proof}
	\begin{align*}
		F_\B(x) &= \big(x-\sigma_1(\B)\big) \times\dots\times\big(x-\sigma_d(x)\big)\\
		& = x^d - \big( \sigma_1(\B)+\dots + \sigma_d(\B) \big) x^{d-1} + \dots + (-1)^d\big(\sigma_1(\B)\times \dots \times \sigma_d(\B)\big)\\
		&= x^d - Tr(\B)x^{d-1} + \dots + (-1^d)N(\B)
	\end{align*}
\end{proof}
\spac
\textbf{E.g.} $\A = 10^\frac{1}{3}$, $\B = \frac{1+\A+\A^2}{3}$
\spa

$N(\B) = 3$, $Tr(\B) = 1$ and $F_\B (\A) = x^3 -x^2 -3x -3$

\begin{cor}
	If $\B \in \ring_k$ then $N(\B), Tr(\B) \in \Z$.
\end{cor}

\begin{proof}
	$ F_\B \in \Z[x]$
\end{proof}
\spa
\textbf{E.g.} $k = \Q(i)$, $\B = \frac{1+i}{2}$
\spa

$N(\B) = \frac{1+i}{2} * \frac{1-i}{2} = \frac{1+1}{4} = \frac{1}{2} \not \in \Z$
\spa

$ \B \in \ring_k$




\subsection{Integral Bases}

\textbf{E.g.} $\A = 10^{\frac{1}{3}}$, $k = \Q(\A)$ $\Z[\A] \subseteq \ring_k$
\spa

We also found another element $\B \in \ring_k $, $\B = \frac{1+\A+\A^2}{3}$ for $\Z[\A, \B] \in \ring_k$.
\spa

Is this all of $\ring_k$ or are there more?
\spa

Eventually this process will end with the whole of $\ring_k$ which will be proved later.

\begin{defn}
	Suppose $\mathcal{B} = \{b_1, \dots, b_n \} $ is a basis for $k$ as a vector space over $\Q$.
	We call $\B$ an integral basis if:
	\begin{align*}
		\ring_k = Span_\Z(\B = \{x_1b_1 + \dots + x_db_d: x_i \in \Z \})
	\end{align*}
\end{defn}
\begin{theorem}
	There exists an integral basis in $k$ and a method for finding it.
\end{theorem}

\begin{proof}
	(Sketch)

	Start with any basis $\{ b_1,\dots, b_d\}$.
	\spa

	After multiplying $b_i$ by integer, we can assume $b_i \in \ring_k$, $Span_\Z\{b_i\} \subset\Q$
	\spa

	If $Span_\Z \bba \neq \ring_k$ then we can find an element $\ring \in \ring_k$ with $\ring \not \in Span_\Z \B$.
	\spa

	Replace some $b_i$ by $\ring$ to get a new basis $\mathcal{C}$ and this new basis is "smaller" than $\mathcal{B}$ but we need to explain what smaller means.
\end{proof}


































\end{document}
 

