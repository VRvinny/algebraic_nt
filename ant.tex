\documentclass[11pt]{article}

\usepackage[left=3cm, right=3cm, top=3cm, bottom=3cm]{geometry}
\usepackage{amsmath}
\usepackage{amsfonts}
\usepackage{amsthm}
\usepackage{ragged2e}
%crossed out elements
\usepackage{cancel}
\usepackage{amssymb}
%lightning symbol
\usepackage{wasysym}
%position tables
\usepackage{placeins}
%use right curly braces
%\usepackage{mathtools}
%place text under sigmas and pi
\usepackage{mathtools}
%kappa%
\newenvironment{rcurly}{\left.\begin{aligned}}{\end{aligned}\right\rbrace}

\begin{document}

\numberwithin{equation}{subsection}
\newtheorem{theorem}{Theorem}[section]
\newtheorem{definition}[theorem]{Defintion}
\newtheorem{proposition}[theorem]{Proposition}
\newtheorem{corollary}[theorem]{Corollary}
\newtheorem{lemma}[theorem]{Lemma}
%\newcommand{\R}{\mathbb{R}}
%\newcommand{\R}{\mathbb{Z}}
%\newcommand{\R}{\mathbb{C}}
%\begin{theorem}
%\begin{theorem}

\title{Algebraic Number Theory}
\author{Vinesh Ramgi}
	%\date{}
	\maketitle


\newpage
	\begin{abstract}

		\centering{What did the number theorist say as he drowned?}\\[1em]
		Log, log, log, log....\\[2em]

		For an up to date version of this pdf, check my GitHub :)\\[0.5em]
https://github.com/vrvinny/nuber-theory

	\end{abstract}
\newpage
\tableofcontents{}
\newpage
	\section{Introduction/Review}
	\subsection{Introduction}
	Number Theory is the theory of the ring $\mathbb{Z}$ and other related rings. A ring (in this course) is a set $R$ with two binary operations $+$ and $*$ such that:








\end{document}
 

